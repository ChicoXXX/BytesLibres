%%% LaTeX Template: Two column article
%%%
%%% Source: http://www.salazarysanchez.com/
%%% Usen libremente este template, el cual se tomo de http://www.howtotex.com/.
%%% Fecha: Abril 2013

%%% Preamble
\documentclass[	DIV=calc,%
							paper=letterpaper,%
							fontsize=13pt,%
							twocolumn]
{article}	 					% KOMA-article class

\usepackage{lipsum}						% Package to create dummy text

% \usepackage[english]{babel}					% English language/hyphenation
\usepackage[protrusion=true,expansion=true]{microtype}		% Better typography
\usepackage{amsmath,amsfonts,amsthm}				% Math packages
\usepackage[pdftex]{graphicx}					% Enable pdflatex
\usepackage[svgnames]{xcolor}						% Enabling colors by their 'svgnames'
\usepackage[hang, small,labelfont=bf,up,textfont=it,up]{caption}	% Custom captions under/above floats
\usepackage{epstopdf}							% Converts .eps to .pdf
\usepackage{subfig}							% Subfigures
\usepackage{booktabs}							% Nicer tables
\usepackage{fix-cm}							% Custom fontsizes
\usepackage[utf8x]{inputenc}    % Para permitir la ñ y acentos
\usepackage[spanish]{babel}     % Para permitir la ñ y acentos



%%% Custom sectioning (sectsty package)
\usepackage{sectsty}					% Custom sectioning (see below)
\allsectionsfont{%					% Change font of al section commands
	\usefont{OT1}{phv}{b}{n}%			% bch-b-n: CharterBT-Bold font
	}

\sectionfont{%						% Change font of \section command
	\usefont{OT1}{phv}{b}{n}%			% bch-b-n: CharterBT-Bold font
	}



%%% Headers and footers
\usepackage{fancyhdr}					% Needed to define custom headers/footers
	\pagestyle{fancy}				% Enabling the custom headers/footers
\usepackage{lastpage}	

% Header (empty)
\lhead{}
\chead{}
\rhead{}
% Footer (you may change this to your own needs)
\lfoot{\footnotesize \texttt{http://gulag.org.mx} }
\cfoot{}
\rfoot{\footnotesize página \thepage\ de \pageref{LastPage}}	% "Page 1 of 2"
\renewcommand{\headrulewidth}{0.0pt}
\renewcommand{\footrulewidth}{0.4pt}



%%% Creating an initial of the very first character of the content
\usepackage{lettrine}
\newcommand{\initial}[1]{%
     \lettrine[lines=3,lhang=0.3,nindent=0em]{
     				\color{DarkGoldenrod}
     				{\textsf{#1}}}{}}



%%% Title, author and date metadata
\usepackage{titling}						% For custom titles

\newcommand{\HorRule}{\color{DarkGoldenrod}%			% Creating a horizontal rule
									  	\rule{\linewidth}{1pt}%
										}

\pretitle{\vspace{-30pt} \begin{flushleft} \HorRule 
				\fontsize{50}{50} \usefont{OT1}{phv}{b}{n} \color{DarkRed} \selectfont 
				}
\title{Primeros pasos con Git y GitHub}				% Title of your article goes here
\posttitle{\par\end{flushleft}\vskip 0.5em}

\preauthor{\begin{flushleft}
					\large \lineskip 0.5em \usefont{OT1}{phv}{b}{sl} \color{DarkRed}}
\author{Osvaldo R. Salazar S.}			% Author name goes here
\postauthor{\footnotesize \usefont{OT1}{phv}{m}{sl} \color{Black} 
 							% Institution of author
					\par\end{flushleft}\HorRule}

\date{}							% No date



%%% Begin document
\begin{document}
\maketitle
\thispagestyle{fancy} 			% Enabling the custom headers/footers for the first page 
% The first character should be within \initial{}
\initial{E}\textbf{l objetivo de este articulo es mostrar como trabajar con GitHub y por lógica, Git.}

\section*{Pero antes, un poco de teoría}
GitHub [1] es una forja para alojar proyectos utilizando el sistema de control de versiones Git. Utiliza el framework Ruby on Rails por GitHub, Inc. (anteriormente conocida como Logical Awesome).

El código se almacena de forma pública, sin costo; aunque también se puede hacer de forma privada, creando de esta manera una cuenta de pago.

Git [2] es un software de control de versiones diseñado por Linus Torvalds, pensando en la eficiencia y la confiabilidad del mantenimiento de versiones de aplicaciones cuando estas tienen un gran número de archivos de código fuente. Al principio, Git se pensó como un motor de bajo nivel sobre el cual otros pudieran escribir la interfaz de usuario o front end como Cogito o StGIT. Sin embargo, Git se ha convertido desde entonces en un sistema de control de versiones con funcionalidad plena. Hay algunos proyectos de mucha relevancia que ya usan Git, en particular, el grupo de programación del núcleo Linux.

El mantenimiento del software Git está supervisado por Junio Hamano desde 2009, quien recibe contribuciones al código de alrededor de 280 programadores. 

\section*{Ahora, manos al teclado :)}
Empezare por decir que estare trabajando con Debian Squeeze, pero puedes usar la distribución que te agrade (recuerda: Libertad ;) ) solo vas a tener que ejecutar los comando equivalentes.

Empezaremos por instalar git en nuestro equipo con el comando:

\begin{verbatim}
$sudo apt-get install git
\end{verbatim}


A continuación procederemos a configurar nuestros datos personales, mismos que se reflejaran cuando envíemos nuestras contribuciones; lo haremos con los siguientes comandos:

\begin{verbatim}
$git config --global user.name TU-NOMBRE
\end{verbatim}

\begin{verbatim}
$git config --global user.email TU-EMAIL
\end{verbatim}

Muy bien, ahora bajaremos de GitHub un proyecto... ¿que tal el presente artículo?... y lo haremos con el siguiente comando:

\begin{verbatim}
$git clone https://github.com/ChicoXXX/BytesLibres.git
\end{verbatim}

Con el anterior comando tendremos un directorio en nuestro equipo, para este ejemplo, llamado BytesLibres al cual entraremos:

\begin{verbatim}
$cd BytesLibres
\end{verbatim}

Y podemos empezar a contribuir con el presente articulo con el siguiente comando:

\begin{verbatim}
$nano 2013/04-abril/"Primeros pasos con Git y GitHub"
\end{verbatim}

¿Y qué creen que sigue? ... ¡Adivinaron! tomar aire y escribir :-D

Al terminar de escribir, grabamos con Control+o y salimos con Control+x. Acto seguido podemos ver el estatus de nuestra copia local con:

\begin{verbatim}
$git status
\end{verbatim}

Lo que veremos son los archivos que se han modificado. Los cambios que acabamos de hacer los agregaremos con:

\begin{verbatim}
$git add .
\end{verbatim}

Vemos a continuación nuevamente el estatus con el comando que se indico líneas arriba para despues hacer un commit:

\begin{verbatim}
$git commit -m "INDICO LOS CAMBIOS QUE REALICE."
\end{verbatim}

Es de notar el parametro -m y el texto que va entre comillas, ya que con esto se grabara en el log un texto indicando el cambio que se realizo. Ahora subiremos los cambios a GitHub.com con:

\begin{verbatim}
$git push origin master
\end{verbatim}

Con lo cua nos pedira nuestro usuario y password de GitHub, mismos que podemos obtener en caso de no tenerlos entrando a www.github.com poner nuestro usuario, email, password y dar click en "Sign up for free".

Y de esta manera ya empezamos a colaborar con este artículo, y no se preocupen, todo queda en los logs y podemos observar quien a colaborado y con que.

Podemos bajar las colaboraciones de otras personas con el comando:

\begin{verbatim}
$git pull
\end{verbatim}

Esta el la forma básica de trabajar, para la siguiente entrega (a menos que alguien colabore en el presente artículo) mostraremos como configurar el acceso por ssh y como eliminar archivos.

\section*{Para terminar}
Espero que esta forma de trabajar sea de buen agrado para muchos, así como me ha gustado, y colaboren con este y con otros proyectos, es interesante ver un buen proyecto de código abierto que trabaje para la comunidad y, no se... Empresas, emprendedor@s, amig@s y colaborador@s: ¿Incluir en nuetro CV el trabajo realizado en GitHub? ¿Qué opinan?

\section*{Referencias:}
[1] https://es.wikipedia.org/wiki/GitHub
[2] https://es.wikipedia.org/wiki/Git
[3] https://help.github.com/articles/set-up-git
[4] http://www.github.com


\end{document}